\documentclass[journal=jpcbfk,manuscript=article]{achemso}

\usepackage{booktabs}
\usepackage{amssymb}
\usepackage{amsmath}
\everymath{\displaystyle}
\usepackage[brazilian,english]{babel}
\usepackage{graphicx}
\usepackage{dcolumn}
\usepackage{hyperref}
\usepackage{float}
\hypersetup{colorlinks=true,linkcolor=black,citecolor=blue,urlcolor=blue}
\usepackage[T1]{fontenc} 
\usepackage[utf8]{inputenc}
\usepackage{makecell}
\usepackage{color} % Include colors for document elements
\usepackage{easyReview}



\title{PRIMoRDiA Meets EasyHybrid3: Unlocking the Exploration of the Electronic Structure of Biomolecules Through Hybrid QC/MM Simulations On the Fly.}

\author{Igor Barden Grillo}
\affiliation{Departament of Chemistry of Federal University of Paraíba, João Pessoa - PB, Brazil}

\author{José Fernando Ruggiero Bachega}
\affiliation{Department of Pharmacosciences, Federal University of Health Sciences of Porto Alegre, Centro Histórico, Porto Alegre- RS,90050-170, Brazil.} 

\author{Martin J. Field}
\affiliation{Laboratoire de Chimie et Biologie des M\'{e}taux, UMR5249, Universit\'{e} Grenoble I, CEA, CNRS, 17 avenue des Martyrs, 38054 Grenoble Cedex 9, France  }
\affiliation{Theory Group, Institut Laue-Langevin, 71 avenue des Martyrs CS 20156, 38042 Grenoble Cedex 9, France}

\author{Larissa Pereira}
\affiliation{Departament of Chemistry of Federal University of Paraíba, João Pessoa - PB,  Brazil}

\author{Gerd Bruno Rocha}
\affiliation{Departament of Chemistry of Federal University of Paraíba, João Pessoa - PB,  Brazil}



\begin{document}
	\maketitle
	\newpage
	\begin{abstract}
		Hybrid simulations are powerful computational chemistry tools to obtain structural and thermodynamics information about enzymatic catalysis reactions. In addition to such relevant applications, these calculations also provide the resolved electronic structure, which is require to estimate the electronic energy in the SCF process. Recently, our group is using such information to obtain several quantum chemical descriptors to profile the steering forces underlying biological processes.In our’s group recent main recent publications we show: how semiempirical Hamiltonian could be used to obtain the Fukui functions from polypeptides with modified descriptors, applying them to ligand-binding process; to profile enzymatic catalysis reaction coordinates; to study the main interactions on the active site of main protease SARS-COV-II, comparing with from the SARS-COV-I.	The main goal of this work is to present our software PRIMoRDiA, which is a new application written to deal efficiently with large outputs of quantum chemical programs and to calculate more than thirty global and local quantum chemical descriptors. The new version of the program, is planned to work directly with the electronic structure obtained with pDynamo software through EasyHybrid 3.0 interface, allowing such theoretical analysis on the fly for hybrid QC/MM simulations.
	\end{abstract}
	\section{Introduction}
		The developing of a computational protocols such the quantum chemical/quantum mechanical hybrid schemes, allowed the study of complex process that occurs in large biomolecular systems. This turned the Hybrid QC/MM methods the state of art and it comes to the energetic and structural studies of enzymatic catalysis. Although, in the great majority of those simulations the resolved electronic structure is trow away, i.e. the wave function information is not utilized even though the computational cost to obtain it  already was paid. 
		
		PRIMoRDiA software is a new application released to the community prepared to deal with large output files from quantum chemical packages, transforming the electronic structure information in quantum chemical descriptors, global and local, that describes the reactivity propensity and/or provide some electronic properties that can be to used to train correlation models. 
		
		%about the applications of PRIMoRDiA with QC/MM
		We already used PRIMoRDiA in some studies for profile and characterize reaction paths of enzymatic reactions, from trajectories generated with QC/MM simulations. 
		
		%Other applucations of these electronic strucutre studies
		
		%The goal of the study
		
		%Description of the RTA system	
	
	\begin{figure}[H]
		\includegraphics[width=5in]{rta_mechanism}
		\caption{ }
		\label{fig:mechanism}			
	\end{figure}


	\section{Methods}
	
	the system was prepared for QC/MM simulations with the following steps: i) parametrization of the cofactor NADP; ii) parametrization of the substrate; ii) solvation of the complex, ions addition to neutralize the total charge and protein parameters generation with AMBER force field; iv) GROMACS energy minimization of the entire system. After these steps, the force field parameters and coordinates files were loaded in the pDynamo library, where the system were prepared, QC/MM atoms, and fixed atoms were defined and the further simulations were performed.
	%put figure with the flow chart
	
	\begin{figure}[H]
		\includegraphics[width=5in]{rta_mechanism}
		\caption{ }
		\label{fig:mechanism}			
	\end{figure}
	
	\subsection{Structure Preparation and Parametrization}
	
	%pick the parametrization details from the larissa dissertation
	
	the library files for tleap were generated with parmchk2 from ambertools. The glucose-6-phosphate parameters were generated with am1-bcc charge scheme using antechamber program considering -2 of formal charge. We used the amberff99SB force filed for the protein atoms, tip3p model to build a solvation cubic box respecting the 12\AA\ between the surface of protein complex until the box end. Also, seven sodium ions were added to neutralize the total system charge. 
	
	We build the complex parameters and coordinates using tleap program, the parameters files were converted to GROMACS format using the Parmed library for Python 3, for energy minimization. We performed this minimization using steepest descent algorithm, periodic boundary conditions in all dimensions, 50000 maximum number of steps and 1000 kJ/mol/nm force tolerance and 0.01 of energy step.
	
	\subsection{QC/MM Simulations Set Up}
	
	%detaisl about the QC/MM region picking
	
	%figure with the QC regionsm first and expanded  
	
	The subsequent coordinates and the force field parameters were loaded and treated by the library classes and functions of pDynamo\cite{field2008pdynamo} and EasyHybrid\cite{bachega2013gtkdynamo}. Firstly, the box was set to a sphere, taking the hydride (H1) as the center and including all atoms of the residues with up to 30 \AA\ of distance. The atoms of the residues within 22 \AA\ from the center was considered as mobile atoms and beyond as fixed, i.e., they only interact with the other atoms but without to get their coordinates updated. This avoids unnecessary gradient calculations without losing the effect of this outer layer. The distances were chosen to consider the NADP molecule present in the other biding pocket, as well to provide a minimum water molecules layer. After that, a geometry optimization was performed using pDynamo conjugated gradient minimizer with a 0.1 kJ.mol$^{-1}.$\AA $^{-1}$ for root mean square tolerance. 
	
	The most critical definition is the atoms that will be treated with the quantum chemistry potential. An unnecessary inclusions of atoms, in respect to the description of the reactions, could lead to unpractical simulations times for energy calculations employing DFT functionals or further free energy evaluations. Thus, we included only the most important moieties to the proposed catalysis mechanism, which includes: the nicotimide ring atoms from NADP; all glucose-6-phosphate atoms; the imidazole ring atoms of the catalytic histidine-337 and the near histidine-175; amine group of the lysine-145 and the carbon bind with the nitrogen, which is a residue very near the oxygen O1 of the substrate; and finally the carboxylic oxygens and carbon of the aspartate-174, that could activate the HIS237. The pruned system and division set up is shown in the \autoref{fig:1}.
	
	\subsection{Reactivity Descriptors}

	\section{Results}
	
	\subsection{QC/MM Molecular Dynamics Sampling}
	
	\subsection{Reactivity Descriptors}
	
	\subsection{Statistical Analysis}
	
	\subsection{Structural and Reactivity Analysis}
	
	\section{Conclusions}
	
	
	
\bibliography{refs}
	
	
\end{document}